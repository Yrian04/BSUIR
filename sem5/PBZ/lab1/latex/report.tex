\documentclass[a4paper,8pt]{article}
\usepackage{graphicx}
\usepackage[utf8]{inputenc}
\usepackage[russian]{babel}
\usepackage[T2A]{fontenc}
\usepackage{float}
\usepackage{geometry}
\usepackage{booktabs}
\geometry{
  a4paper,
  top=25mm,
  right=20mm,
  bottom=25mm,
  left=20mm
}

\title{Отчет по запросам SQL}
\author{}
\date{}

\begin{document}

\begin{titlepage}
  \begin{center}
    Министерство образования Республики Беларусь\\[1em]
    Учреждение образования\\
    БЕЛОРУССКИЙ ГОСУДАРСТВЕННЫЙ УНИВЕРСИТЕТ \\
    ИНФОРМАТИКИ И РАДИОЭЛЕКТРОНИКИ\\[8em]

    \begin{minipage}{\textwidth}
      \begin{flushleft}
        \begin{tabular}{ l l }
          Факультет & информационных технологий и управления\\
          Кафедра   & интеллектуальных информационных технологий
        \end{tabular}
      \end{flushleft}
    \end{minipage}\\[8em]

    {Отчёт}\\
    {по лабораторной работе №1}\\
    {по дисциплине <<Проектирование баз знаний>>}\\[8em]

    \vspace{2em}
    \begin{tabular}{ p{0.65\textwidth}p{0.25\textwidth} }
      Студент гр. 221701 & Е.\,Д.~Глёза \\
      Проверил & М.\,Г.~Соколович \\
    \end{tabular}
    
    \vfill
    {\normalsize Минск 2024}
  \end{center}
\end{titlepage}

\section{Задание 1}

\subsection{Получить полную информацию обо всех преподавателях.}
\begin{verbatim}
select * from teachers;
\end{verbatim}
\begin{table}[H]
\centering
\begin{tabular}{rlllll}
\toprule
id & second_name & position & department & specialization & number \\
\midrule
110 & Петров & Ассистент & Экономики & Международная экономика & 324 \\
221 & Фролов & Доцент & ЭВМ & АСОИ, ЭВМ & 487 \\
222 & Костин & Доцент & ЭВМ & ЭВМ & 543 \\
225 & Бойко & Профессор & АСУ & АСОИ, ЭВМ & 112 \\
430 & Глазов & Ассистент & ТФ & СД & 421 \\
\bottomrule
\end{tabular}

\caption{Результаты запроса 1.1}
\end{table}

\subsection{Получить полную информацию обо всех студенческих группах на специальности ЭВМ.}
\begin{verbatim}
select * from stud_groups where specialization = "ЭВМ";
\end{verbatim}
\begin{table}[H]
\centering
\begin{tabular}{rlrll}
\toprule
id & name & size & specialization & chief \\
\midrule
7 & Э-15 & 22 & ЭВМ & Сеткин \\
8 & Э-12 & 18 & ЭВМ & Иванова \\
\bottomrule
\end{tabular}

\caption{Результаты запроса 1.1}
\end{table}

\subsection{Получить личный номер преподавателя и номера аудиторий, в которых они преподают предмет с кодовым номером 18П.}
\begin{verbatim}
select teacherId, roomNumber from group_subject_teacher where subjectId = 18;
\end{verbatim}
\begin{table}[H]
\centering
\begin{tabular}{rr}
\toprule
teacherId & roomNumber \\
\midrule
225 & 210 \\
225 & 210 \\
225 & 210 \\
\bottomrule
\end{tabular}

\caption{Результаты запроса 1.1}
\end{table}

\subsection{Получить номера предметов и названия предметов, которые ведет преподаватель Костин.}
\begin{verbatim}
select subjectId, subjects.name from group_subject_teacher
join subjects on subjects.id = subjectId
join teachers on teachers.id = teacherId
where teachers.second_name = "Костин";
\end{verbatim}
\begin{table}[H]
\centering
\begin{tabular}{rl}
\toprule
subjectId & name \\
\midrule
12 & Мини ЭВМ \\
17 & СУБД ПК \\
17 & СУБД ПК \\
12 & Мини ЭВМ \\
12 & Мини ЭВМ \\
12 & Мини ЭВМ \\
12 & Мини ЭВМ \\
12 & Мини ЭВМ \\
\bottomrule
\end{tabular}

\caption{Результаты запроса 1.4}
\end{table}

\subsection{Получить номер группы, в которой ведутся предметы преподавателем Фроловым.}
\begin{verbatim}
select groupId from group_subject_teacher
join teachers on teachers.id = teacherId
where teachers.second_name = "Фролов"
group by groupId;
\end{verbatim}
\begin{table}[H]
\centering
\begin{tabular}{r}
\toprule
groupId \\
\midrule
8 \\
7 \\
3 \\
\bottomrule
\end{tabular}

\caption{Результаты запроса 1.5}
\end{table}

\subsection{Получить информацию о предметах, которые ведутся на специальности АСОИ.}
\begin{verbatim}
select * from subjects where specialization = "АСОИ";
\end{verbatim}
\begin{table}[H]
\centering
\begin{tabular}{rlrlr}
\toprule
id & name & hours & specialization & term \\
\midrule
17 & СУБД ПК & 48 & АСОИ & 4 \\
18 & ВКСС & 52 & АСОИ & 6 \\
\bottomrule
\end{tabular}

\caption{Результаты запроса 1.6}
\end{table}

\subsection{Получить информацию о преподавателях, которые ведут предметы на специальности АСОИ.}
\begin{verbatim}
select * from teachers where specialization like '%АСОИ%';
\end{verbatim}
\begin{table}[H]
\centering
\begin{tabular}{rlllll}
\toprule
id & second_name & position & department & specialization & number \\
\midrule
221 & Фролов & Доцент & ЭВМ & АСОИ, ЭВМ & 487 \\
225 & Бойко & Профессор & АСУ & АСОИ, ЭВМ & 112 \\
\bottomrule
\end{tabular}

\caption{Результаты запроса 1.7}
\end{table}

\subsection{Получить фамилии преподавателей, которые ведут предметы в 210 аудитории.}
\begin{verbatim}
select teachers.second_name from group_subject_teacher
join teachers on teachers.id = teacherId
where roomNumber = 210
group by teachers.second_name;
\end{verbatim}
\begin{table}[H]
\centering
\begin{tabular}{l}
\toprule
second_name \\
\midrule
Бойко \\
Петров \\
Костин \\
\bottomrule
\end{tabular}

\caption{Результаты запроса 1.8}
\end{table}

\subsection{Получить названия предметов и названия групп, которые ведут занятия в аудиториях с 100 по 200.}
\begin{verbatim}
select subjects.name, stud_groups.name from group_subject_teacher
join subjects on subjects.id = subjectId
join stud_groups on stud_groups.id = groupId
where roomNumber between 100 and 200
group by subjects.name, stud_groups.name;
\end{verbatim}
\begin{table}[H]
\centering
\begin{tabular}{ll}
\toprule
name & name \\
\midrule
Мини ЭВМ & Э-12 \\
СУБД ПК & Э-12 \\
Мини ЭВМ & АС-9 \\
Мини ЭВМ & С-14 \\
Физика & С-14 \\
Мини ЭВМ & М-6 \\
Мини ЭВМ & АС-8 \\
\bottomrule
\end{tabular}

\caption{Результаты запроса 1.9}
\end{table}

\subsection{Получить пары номеров групп с одной специальности.}
\begin{verbatim}
select first.id, second.id from stud_groups as first
join stud_groups as second on second.specialization = first.specialization;
\end{verbatim}
\begin{table}[H]
\centering
\begin{tabular}{rr}
\toprule
id & id \\
\midrule
4 & 3 \\
3 & 3 \\
4 & 4 \\
3 & 4 \\
8 & 7 \\
7 & 7 \\
8 & 8 \\
7 & 8 \\
10 & 10 \\
12 & 12 \\
17 & 17 \\
\bottomrule
\end{tabular}

\caption{Результаты запроса 1.10}
\end{table}

\subsection{Получить общее количество студентов, обучающихся на специальности ЭВМ.}
\begin{verbatim}
select sum(size) from stud_groups where specialization = "ЭВМ";
\end{verbatim}
\begin{table}[H]
\centering
\begin{tabular}{r}
\toprule
sum(size) \\
\midrule
40.000000 \\
\bottomrule
\end{tabular}

\caption{Результаты запроса 1.11}
\end{table}

\subsection{Получить номера преподавателей, обучающих студентов по специальности ЭВМ.}
\begin{verbatim}
select teacherId from group_subject_teacher
join stud_groups on stud_groups.id = groupId
where stud_groups.specialization = "ЭВМ"
group by teacherId;
\end{verbatim}
\begin{table}[H]
\centering
\begin{tabular}{r}
\toprule
teacherId \\
\midrule
221 \\
222 \\
225 \\
\bottomrule
\end{tabular}

\caption{Результаты запроса 1.12}
\end{table}

\subsection{Получить номера предметов, изучаемых всеми студенческими группами.}
\begin{verbatim}
insert group_subject_teacher value (7, 12, 222, 234);

select id from subjects as s
where not exists (
	(select id from stud_groups)
    except 
    (select sb.groupId from group_subject_teacher as sb where sb.subjectId = s.id) 
);

delete from group_subject_teacher 
where groupId = 7 and subjectId = 12 and teacherId = 222;
\end{verbatim}
\begin{table}[H]
\centering
Здесь ничего нет
\caption{Результаты запроса 1.13}
\end{table}

\subsection{Получить фамилии преподавателей, преподающих те же предметы, что и преподаватель, преподающий предмет с номером 14П.}
\begin{verbatim}
select second_name from teachers as t
where not exists (
	(select gst.subjectId from group_subject_teacher as gst 
    where teacherId in (select gst1.teacherId from group_subject_teacher as gst1 where gst1.subjectId = 14) 
    )
    except 
    (select gst2.subjectId from group_subject_teacher as gst2 where gst2.teacherId = t.id) 
);
\end{verbatim}
\begin{table}[H]
\centering
\begin{tabular}{l}
\toprule
second_name \\
\midrule
Фролов \\
\bottomrule
\end{tabular}

\caption{Результаты запроса 1.14}
\end{table}

\subsection{Получить информацию о предметах, которые не ведет преподаватель с личным номером 221П.}
\begin{verbatim}
select * from subjects
join group_subject_teacher on group_subject_teacher.subjectId = id
where not group_subject_teacher.teacherId = 221;
\end{verbatim}
\begin{table}[H]
\centering
\begin{tabular}{rlrlrrrrr}
\toprule
id & name & hours & specialization & term & groupId & subjectId & teacherId & roomNumber \\
\midrule
12 & Мини ЭВМ & 36 & ЭВМ & 1 & 8 & 12 & 222 & 112 \\
12 & Мини ЭВМ & 36 & ЭВМ & 1 & 4 & 12 & 222 & 112 \\
12 & Мини ЭВМ & 36 & ЭВМ & 1 & 17 & 12 & 222 & 112 \\
12 & Мини ЭВМ & 36 & ЭВМ & 1 & 12 & 12 & 222 & 112 \\
12 & Мини ЭВМ & 36 & ЭВМ & 1 & 10 & 12 & 222 & 210 \\
12 & Мини ЭВМ & 36 & ЭВМ & 1 & 3 & 12 & 222 & 112 \\
17 & СУБД ПК & 48 & АСОИ & 4 & 8 & 17 & 222 & 112 \\
17 & СУБД ПК & 48 & АСОИ & 4 & 7 & 17 & 222 & 241 \\
18 & ВКСС & 52 & АСОИ & 6 & 7 & 18 & 225 & 210 \\
18 & ВКСС & 52 & АСОИ & 6 & 4 & 18 & 225 & 210 \\
18 & ВКСС & 52 & АСОИ & 6 & 3 & 18 & 225 & 210 \\
22 & Аудит & 24 & Бухучет & 3 & 17 & 22 & 110 & 220 \\
22 & Аудит & 24 & Бухучет & 3 & 12 & 22 & 110 & 210 \\
22 & Аудит & 24 & Бухучет & 3 & 10 & 22 & 110 & 210 \\
34 & Физика & 30 & СД & 6 & 17 & 34 & 430 & 118 \\
\bottomrule
\end{tabular}

\caption{Результаты запроса 1.15}
\end{table}

\subsection{Получить информацию о предметах, которые не изучаются в группе М-6.}
\begin{verbatim}
select * from subjects
join group_subject_teacher on group_subject_teacher.subjectId = id
join stud_groups on stud_groups.id = group_subject_teacher.groupId
where not stud_groups.name = "M-6";
\end{verbatim}
\begin{table}[H]
\centering
\begin{tabular}{rlrlrrrrrrlrll}
\toprule
id & name & hours & specialization & term \\
\midrule
17 & СУБД ПК & 48 & АСОИ & 4 \\
18 & ВКСС & 52 & АСОИ & 6 \\
12 & Мини ЭВМ & 36 & ЭВМ & 1 \\
12 & Мини ЭВМ & 36 & ЭВМ & 1 \\
18 & ВКСС & 52 & АСОИ & 6 \\
14 & ПЭВМ & 72 & ЭВМ & 2 \\
17 & СУБД ПК & 48 & АСОИ & 4 \\
18 & ВКСС & 52 & АСОИ & 6 \\
12 & Мини ЭВМ & 36 & ЭВМ & 1 \\
14 & ПЭВМ & 72 & ЭВМ & 2 \\
17 & СУБД ПК & 48 & АСОИ & 4 \\
12 & Мини ЭВМ & 36 & ЭВМ & 1 \\
22 & Аудит & 24 & Бухучет & 3 \\
12 & Мини ЭВМ & 36 & ЭВМ & 1 \\
22 & Аудит & 24 & Бухучет & 3 \\
12 & Мини ЭВМ & 36 & ЭВМ & 1 \\
22 & Аудит & 24 & Бухучет & 3 \\
34 & Физика & 30 & СД & 6 \\
\bottomrule
\end{tabular}

\caption{Результаты запроса 1.16}
\end{table}

\subsection{Получить информацию о доцентах, преподающих в группах 3Г и 8}
\begin{verbatim}
select teachers.* from teachers
join group_subject_teacher on group_subject_teacher.teacherId = teachers.id
join stud_groups on stud_groups.id = group_subject_teacher.groupId
where stud_groups.name in ("3Г", "8");
\end{verbatim}
\begin{table}[H]
\centering
\begin{tabular}{rlllllrrrr}
\toprule
id & second_name & position & department & specialization\\
\midrule
221 & Фролов & Доцент & ЭВМ & АСОИ, ЭВМ \\
221 & Фролов & Доцент & ЭВМ & АСОИ, ЭВМ \\
222 & Костин & Доцент & ЭВМ & ЭВМ \\
222 & Костин & Доцент & ЭВМ & ЭВМ \\
222 & Костин & Доцент & ЭВМ & ЭВМ \\
\bottomrule
\end{tabular}

\caption{Результаты запроса 1.17}
\end{table}

\subsection{Получить номера предметов, номера преподавателей, номера групп, в которых ведут занятия преподаватели с кафедры ЭВМ, имеющих специальность АСОИ.}
\begin{verbatim}
select subjectId, teacherId, groupId from group_subject_teacher
join teachers on teachers.id = teacherId
where teachers.department = "ЭВМ" and specialization like '%АСОИ%';
\end{verbatim}
\begin{table}[H]
\centering
\begin{tabular}{rrr}
\toprule
subjectId & teacherId & groupId \\
\midrule
14 & 221 & 8 \\
14 & 221 & 7 \\
17 & 221 & 3 \\
\bottomrule
\end{tabular}

\caption{Результаты запроса 1.18}
\end{table}

\subsection{Получить номера групп с такой же специальностью, что и специальность преподавателей.}
\begin{verbatim}
select groupId from group_subject_teacher
join stud_groups on stud_groups.id = groupId
join teachers on teachers.id = teacherId
where not locate(stud_groups.specialization, teachers.specialization) = 0
group by groupId;
\end{verbatim}
\begin{table}[H]
\centering
\begin{tabular}{r}
\toprule
groupId \\
\midrule
12 \\
8 \\
7 \\
3 \\
4 \\
17 \\
\bottomrule
\end{tabular}

\caption{Результаты запроса 1.19}
\end{table}

\subsection{Получить номера преподавателей с кафедры ЭВМ, преподающих предметы по специальности, совпадающей со специальностью студенческой группы.}
\begin{verbatim}
select t.id from (select * from teachers where department = "ЭВМ") as t
join group_subject_teacher on group_subject_teacher.teacherId = t.id
join stud_groups on stud_groups.id = group_subject_teacher.groupId
where not locate(stud_groups.specialization, t.specialization) = 0;
\end{verbatim}
\begin{table}[H]
\centering
\begin{tabular}{r}
\toprule
id \\
\midrule
221 \\
221 \\
221 \\
222 \\
222 \\
222 \\
\bottomrule
\end{tabular}

\caption{Результаты запроса 1.20}
\end{table}

\subsection{Получить специальности студенческой группы, на которых работают преподаватели кафедры АСУ.}
\begin{verbatim}
select stud_groups.specialization from stud_groups
join group_subject_teacher on group_subject_teacher.groupId = stud_groups.id
join teachers on teachers.id = group_subject_teacher.teacherId
where teachers.department = "АСУ";
\end{verbatim}
\begin{table}[H]
\centering
\begin{tabular}{l}
\toprule
specialization \\
\midrule
ЭВМ \\
АСОИ \\
АСОИ \\
\bottomrule
\end{tabular}

\caption{Результаты запроса 1.21}
\end{table}

\subsection{Получить номера предметов, изучаемых группой АС-8.}
\begin{verbatim}
select subjects.id from subjects
join group_subject_teacher on group_subject_teacher.subjectId = subjects.id
join stud_groups on stud_groups.id = group_subject_teacher.groupId
where stud_groups.name = "АС-8";
\end{verbatim}
\begin{table}[H]
\centering
\begin{tabular}{r}
\toprule
id \\
\midrule
17 \\
18 \\
12 \\
\bottomrule
\end{tabular}

\caption{Результаты запроса 1.22}
\end{table}

\subsection{Получить номера студенческих групп, которые изучают те же предметы, что и студенческая группа АС-8.}
\begin{verbatim}
select g.id from stud_groups as g
where not exists (
	(
		select s1.id from subjects as s1
		join group_subject_teacher as gst1 on gst1.subjectId = s1.id
		join stud_groups as g1 on g1.id = gst1.groupId
		where g1.name = "АС-8"
    )
    except 
    (
		select s2.id from subjects as s2
		join group_subject_teacher as gst2 on gst2.subjectId = s2.id
		join stud_groups as g2 on g2.id = gst2.groupId
		where g2.id = g.id
    )
);
\end{verbatim}
\begin{table}[H]
\centering
\begin{tabular}{r}
\toprule
id \\
\midrule
3 \\
\bottomrule
\end{tabular}

\caption{Результаты запроса 1.23}
\end{table}

\subsection{Получить номера студенческих групп, которые не изучают предметы, преподаваемые в студенческой группе АС-8.}
\begin{verbatim}
delete from group_subject_teacher 
where groupId = 3 and subjectId = 12;

select distinct gst.groupId from group_subject_teacher as gst
where not exists (
	select gst1.subjectId from group_subject_teacher as gst1
    where gst1.groupId = gst.groupId and gst1.subjectId in (
		select gst2.subjectId from group_subject_teacher as gst2
        join stud_groups as g2 on g2.id = gst2.groupId
        where g2.name = "АС-8"
    )
);

insert group_subject_teacher value (3, 12, 222, 112);
\end{verbatim}
\begin{table}[H]
\centering
Здесь ничего нет
\caption{Результаты запроса 1.24}
\end{table}

\subsection{Получить номера студенческих групп, которые не изучают предметы, преподаваемые преподавателем 430Л.}
\begin{verbatim}
select distinct gst.groupId from group_subject_teacher as gst
where not exists (
	select gst1.subjectId from group_subject_teacher as gst1
    where gst1.groupId = gst.groupId and gst1.subjectId in (
		select gst2.subjectId from group_subject_teacher as gst2
        where gst2.teacherId = 430
    )
);
\end{verbatim}
\begin{table}[H]
\centering
\begin{tabular}{r}
\toprule
groupId \\
\midrule
3 \\
4 \\
7 \\
8 \\
10 \\
12 \\
\bottomrule
\end{tabular}

\caption{Результаты запроса 1.25}
\end{table}

\subsection{Получить номера преподавателей, работающих с группой Э-15, но не преподающих предмет 12П.}
\begin{verbatim}
select gst.teacherId from group_subject_teacher as gst
join stud_groups as g on g.id = gst.groupId
where g.name = "Э-15" and 12 not in (
	select gst1.subjectId from group_subject_teacher as gst1
    where gst1.teacherId = gst.teacherId
);
\end{verbatim}
\begin{table}[H]
\centering
\begin{tabular}{r}
\toprule
teacherId \\
\midrule
221 \\
225 \\
\bottomrule
\end{tabular}

\caption{Результаты запроса 1.26}
\end{table}

\section{Запросы}

\subsection{Получить номера поставщиков, поставляющих одну и ту же деталь для всех проектов.}
\begin{verbatim}
select supplierId from suppliers_parts_projects
group by supplierId
having count(distinct partId) = 1;
\end{verbatim}
\begin{table}[H]
\centering
\begin{tabular}{r}
\toprule
supplierId \\
\midrule
1 \\
4 \\
\bottomrule
\end{tabular}

\caption{Результаты запроса 2.31}
\end{table}

\subsection{Получить номера деталей, поставляемых для всех проектов, обеспечиваемых поставщиком из того же города, где размещен проект.}
\begin{verbatim}
select partId from suppliers_parts_projects
join suppliers on supplierId = suppliers.id
join projects on projectId = projects.id
where suppliers.location = projects.location
group by partId;
\end{verbatim}
\begin{table}[H]
\centering
\begin{tabular}{r}
\toprule
partId \\
\midrule
4 \\
2 \\
5 \\
\bottomrule
\end{tabular}

\caption{Результаты запроса 2.12}
\end{table}

\subsection{Получить цвета деталей, поставляемых поставщиком П1.}
\begin{verbatim}
select color from parts
join suppliers_parts_projects on suppliers_parts_projects.partId = id
where suppliers_parts_projects.supplierId = 1
group by color;
\end{verbatim}
\begin{table}[H]
\centering
\begin{tabular}{l}
\toprule
color \\
\midrule
Красный \\
\bottomrule
\end{tabular}

\caption{Результаты запроса 2.20}
\end{table}

\subsection{Получить номера проектов, обеспечиваемых по крайней мере одним поставщиком не из того же города.}
\begin{verbatim}
select projectId from suppliers_parts_projects
join suppliers on suppliers.id = supplierId
join projects on projects.id = projectId
where suppliers.location != projects.location
group by projectId;
\end{verbatim}
\begin{table}[H]
\centering
\begin{tabular}{r}
\toprule
projectId \\
\midrule
1 \\
2 \\
3 \\
4 \\
5 \\
6 \\
7 \\
\bottomrule
\end{tabular}

\caption{Результаты запроса 2.13}
\end{table}

\subsection{Получить номера поставщиков со статусом, меньшим чем у поставщика П1.}
\begin{verbatim}
select id from suppliers
where status < (select status from suppliers as sub where sub.id = 1);
\end{verbatim}
\begin{table}[H]
\centering
\begin{tabular}{r}
\toprule
id \\
\midrule
2 \\
\bottomrule
\end{tabular}

\caption{Результаты запроса 2.24}
\end{table}

\subsection{Получить полную информацию обо всех проектах в Лондоне.}
\begin{verbatim}
select * from projects
where location = "Лондон";
\end{verbatim}
\begin{table}[H]
\centering
Здесь ничего нет
\caption{Результаты запроса 2.2}
\end{table}

\subsection{Получить все такие тройки "номера поставщиков-номера деталей-номера проектов", для которых выводимые поставщик, деталь и проект размещены в одном городе.}
\begin{verbatim}
select supplierId, partId, projectId from suppliers_parts_projects
join suppliers on suppliers.id = supplierId
join parts on parts.id = partId
join projects on projects.id = projectId
where suppliers.location = parts.location and parts.location = projects.location;
\end{verbatim}
\begin{table}[H]
\centering
Здесь ничего нет
\caption{Результаты запроса 2.6}
\end{table}

\subsection{Получить все пары названий городов, для которых поставщик из первого города обеспечивает проект во втором городе.}
\begin{verbatim}
select suppliers.location, projects.location from suppliers_parts_projects
join suppliers on suppliers.id = supplierId
join projects on projects.id = projectId
group by suppliers.location, projects.location;
\end{verbatim}
\begin{table}[H]
\centering
\begin{tabular}{ll}
\toprule
location & location \\
\midrule
Москва & Минск \\
Москва & Таллин \\
Таллин & Минск \\
Таллин & Таллин \\
Таллин & Псков \\
Таллин & Москва \\
Таллин & Саратов \\
Минск & Псков \\
Минск & Москва \\
Киев & Таллин \\
Киев & Псков \\
Киев & Москва \\
\bottomrule
\end{tabular}

\caption{Результаты запроса 2.11}
\end{table}

\subsection{Получить все такие тройки "номера поставщиков-номера деталей-номера проектов", для которых никакие из двух выводимых поставщиков, деталей и проектов не размещены в одном городе.}
\begin{verbatim}
select supplierId, partId, projectId from suppliers_parts_projects
join suppliers on suppliers.id = supplierId
join parts on parts.id = partId
join projects on projects.id = projectId
where not (
suppliers.location = parts.location or parts.location = projects.location
or suppliers.location = projects.location);
\end{verbatim}
\begin{table}[H]
\centering
\begin{tabular}{rrr}
\toprule
supplierId & partId & projectId \\
\midrule
2 & 3 & 1 \\
2 & 3 & 3 \\
2 & 3 & 4 \\
2 & 3 & 5 \\
2 & 3 & 6 \\
2 & 3 & 7 \\
3 & 3 & 1 \\
4 & 6 & 3 \\
5 & 2 & 2 \\
5 & 2 & 4 \\
5 & 5 & 5 \\
5 & 5 & 7 \\
5 & 6 & 2 \\
5 & 1 & 2 \\
5 & 3 & 4 \\
5 & 4 & 4 \\
5 & 5 & 4 \\
5 & 6 & 4 \\
\bottomrule
\end{tabular}

\caption{Результаты запроса 2.8}
\end{table}

\subsection{Получить номера проектов, для которых среднее количество поставляемых деталей Д1 больше, чем наибольшее количество любых деталей, поставляемых для проекта ПР1.}
\begin{verbatim}
select id from projects
join suppliers_parts_projects on suppliers_parts_projects.projectId = id
where suppliers_parts_projects.partId = 1
group by id
having avg(number) > (select max(a.part_num) from (
	select sum(sub.number) as part_num from suppliers_parts_projects as sub
    where sub.projectId = 1
    group by sub.partId) as a
    );
\end{verbatim}
\begin{table}[H]
\centering
Здесь ничего нет
\caption{Результаты запроса 2.26}
\end{table}

\end{document}



